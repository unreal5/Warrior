\documentclass[
    %blackwhite
]{dragonbane-supplement}
%% ================ TITLE ================ %%
\title{Dragonbane Supplement}
\subtitle{A LaTeX-Template for Dragonbane}
\author{Sibling Dex}
\version{2 (2025-05-15)}
\credits{
    Many thanks to Fahien for feedback and code contribution.
    
    \href{https://www.deviantart.com/esther-sanz/art/Old-Scroll-Texture-II-114214631}{"Old Scroll Texture"} by Esther Sanz(\href{http://creativecommons.org/licenses/by/3.0/}{CC BY 3.0})

    "The Jabberwocky" by John Tenniel (Public Domain).
    
    This game supplement was created under Fria Ligan AB’s \href{https://freeleaguepublishing.com/wp-content/uploads/2023/11/Dragonbane-License-Agreement.pdf}{\emph{Dragonbane Third Party Supplement License}} to be used with the core rules of \textit{Dragonbane}.
    
    This game supplement is neither affiliated with, sponsored, or endorsed by Fria Ligan AB.
}
%% ++++++++++++++++ Code ++++++++++++++++++++ %%
\lstdefinestyle{styleCXX}{
	language = C++,  
	backgroundcolor=\color{blue!3!white}, 
	%basicstyle = \footnotesize,  
	basicstyle      =   \zihao{-5}\ttfamily,
	numberstyle     =   \zihao{-5}\ttfamily,   
	%breakatwhitespace = false,    
	basewidth       =   0.5em,    
	breaklines = true,                 
	captionpos = b,                    
	commentstyle = \color{green!60!black}\bfseries,
	%extendedchars = false,             
	frame =shadowbox, 
	framerule=0.5pt,
	%frameround = fttt,
	keepspaces=true,
	keywordstyle=\color{blue}\bfseries, % keyword style
	otherkeywords={string}, 
	numbers=left, 
	numbersep=5pt,
	numberstyle=\tiny\color{gray},
	rulecolor=\color{black},         
	%showspaces=false,  
	%showstringspaces=false, 
	%showtabs=false,    
	%stepnumber=1,         
	stringstyle=\color{blue},        % string literal style
	tabsize=2,          
	columns         =   fixed,
	flexiblecolumns,                   
}

\newcommand{\ci}[1]{\lstinline[language=c++,breaklines=true,basicstyle=\color{blue}\ttfamily]|#1|}
%% ================ DOCUMENT ================ %%
\begin{document}
\maketitle

\partcolor{DragonRed} % Can be used to change the accent color
\partimage{img/Jabberwocky.jpg}
\part{Instructions}

\partcolor{DemonGreen} % The default accent color
\chapter{文档类特征}

\begin{segment}

\begin{quote}
    "Quotes are useful to provide a short and flavourful introduction to a topic at the top of a new section or chapter."

    --The Author (always give credit)
\end{quote}

\subsection{Parts}
\texttt{\textbackslash part} 命令会生成一个整页的标题,并预留图片空间,如本页前所示。使用 \texttt{\textbackslash partimage} 命令可以设置该页使用的图片。部分(Part)也是使用 \texttt{\textbackslash partcolor} 命令设置许多类元素中强调色的好层级。

\subsection{chapters}
\texttt{\textbackslash chapter} 命令会生成一个大型、强调色、精美的标题,如本页所示。除了装饰外,章节标题看起来与其他由 \texttt{\textbackslash segment} 环境设置的小节标题类似。

\end{segment}

\begin{segment}[Segments]
\texttt{\textbackslash segment} 环境会生成一个两栏布局。这比 \texttt{\textbackslash twocolumn} 选项更容易创建包含许多跨栏元素(如表格)的两栏文档。请记得结束你的分段,否则会出现错误信息。

\paragraph{带标题:}
分段可以带一个可选参数,用于生成如上所示的绿色跨栏 \texttt{\textbackslash section} 标题。这是生成节标题的主要方式,当然你也可以使用 \texttt{\textbackslash section} 命令。但那样的话,请确保将其放在任何 \texttt{\textbackslash segment} 环境\emph{之外},否则会出现位置奇怪的单栏标题。

\paragraph{无标题:}
可选参数也可以省略,这样就不会生成标题,如下表之后所示。

\subsection{小节}
\texttt{subsection} 命令会生成如本段上方所见的标准加粗、左对齐的小节标题。用它们将较长的文本分割成更小的部分,方便读者阅读和定位。

\subsection{段落}
本类文档中最小的定义分级是 \texttt{paragraph}。该命令会生成一个行内加粗标题,适合用于未编号的条目列表,这些条目太长,不适合放在实际的列表环境中。

\paragraph{这是一个段落:}
一种不错的用法是在段落标题末尾加上冒号,如这里所示


\subsection{Lists}
You can use a normal \texttt{itemize} environment and \texttt{\textbackslash item} commands inside a list to produce a simple list entry.

\begin{itemize}
    \bolditem{Bold Item:} To add a bold keyword to the beginning of an item, use the \texttt{\textbackslash bolditem\{...\}} command.
    \coloritem{Color Item:} To make the bold keyword green, use the \texttt{\textbackslash coloritem\{...\}} command. This is often used in lists that list the important aspects of a location.
    \secretitem{Secret Item:} You can make the keyword red and italic by using the \texttt{\textbackslash secretitem\{...\}} command. This is often used to list a secret or hidden feature of a location.
    \coloritem[blue]{Color Item:}  You can make the keyword a custom color by using the \texttt{\textbackslash coloritem[color]\{...\}} command and specify the color in the square bracket.
\end{itemize}
\end{segment}


%% ================================================================ %%

\begin{segment}[Package Options]
\subsection{Generic Options}
You can use all class options available to the standard \texttt{report} document class included in LaTeX.

\paragraph{Papersize:} You can use the class with paper sizes other than A4, but be sure to keep the \texttt{textwidth} at 16cm, otherwise the headers will get uncentered.

\subsection{Black and White}
You can use the \texttt{blackwhite} option in the \texttt{documentclass} command to generate the document in black and white colours. When doing so, you have to manually set monochrome images, the option cannot change included images.
\end{segment}

%% ================================================================ %%

\begin{segment}[Boxes]

There are three types of box provided by this class: the Demonbox, Dragonbox and Emptybox. These are special environments that can be used to highlight special rules or important information in a compact way. These boxes are not floats but are placed as part of the text. Therefore, they can be placed both inside a \texttt{segment} environment, to produce a one-column wide box, or outside, to create a two-column spanning box.

\begin{dragonbox}{Dragonbox}
\begin{itemize}
    \item \textbf{These Rules Are:} Obligatory
\end{itemize}
    This is a \texttt{dragonbox}, it can be used to highlight important information in a compact and noticeable way.
    
    It can be used, for example, to typeset a Heroic Ability. In that case, you can use an \texttt{itemize} list, as above, to note the Willpower cost for the ability.
\end{dragonbox}

\begin{emptybox}{Emptybox}
    This is an \texttt{emptybox}, it features the same heading as a \texttt{demonbox} but not the coloured background. It is used in the Tablebox environment but can also be used by itself.
\end{emptybox}

\begin{demonbox}{Demonbox}
\begin{itemize}
    \item \textbf{These Rules Are:} Optional
\end{itemize}
    This is a \texttt{demonbox}, they are used to add information about optional rules.

    The \texttt{demonbox} is fully coloured in the current accent colour and can be a drain on printer toner or ink. It is generally a good approach to use these boxes sparingly. A restrained use of boxes in general also prevents the layout from looking cluttered.
\end{demonbox}
\end{segment}


\begin{dragonbox}{Wide Boxes}
\begin{segment}
Both \texttt{dragonbox} and \texttt{demonbox} can be used outside a \texttt{segment} environment to make it span the whole page width. When using text inside a wide box, it is good practice to use a \texttt{segment} environment inside the box to get a two column layout in the box and prevent overly long lines. 

\subsection{Subsection}
Lower level headings such as \texttt{subsection} and \texttt{paragraph} can be used inside boxes.

\paragraph{Paragraph:}
Using these headings can help make the text inside a box more ordered and provide a better overview.
\end{segment}
\end{dragonbox}


\begin{tablebox}{Tablebox}
\begin{tabulary}{\linewidth}{ c l c L }
    \textbf{DICE} & \textbf{LABEL} & \textbf{ALIGNMENT} & \textbf{DESCRIPTION} \\
    \hline
    1 & Dice & Center (c) & If you want a table to be rollable, use the first column as the die or dice column. Give it a header denoting the die/dice used, and number the rows. \\
    2 & Label & Left (l) & The first or second column of a table should be the label of the entry. This gives a short and meaningful name to the entry in the row. \\
    3 & Score & Center (c) & You can add several narrower columns for short, standardized scores, such as price, availability, durability, etc. \\
    4 & Description & Left (L) & The typically last column in a table is a longer description of the entry. Use a breaking alignment for this, so the description can be more than one line. \\
\end{tabulary}
\end{tablebox}
\chapter{创建项目}
创建项目\texttt{Crunch},添加\texttt{GameModeBase}的派生类\texttt{CGameMode}。
\section{Gameplay Effect}
\begin{segment}
GE用于修改属性。
\begin{itemize}
    \coloritem[red]{Duration Policy:} 
    	\begin{itemize}
    	\bolditem{Instant:}立即应用\textbf{一次}。
    	\bolditem{Infinite:}一直以指定的间隔(Period)起作用,需要手工移除。
    	\bolditem{Has Duration:}类似Infinite,但有时长,时间到了自动移除。
    	\end{itemize}
    \coloritem{Color Item:} To make the bold keyword green, use the \texttt{\textbackslash coloritem\{...\}} command. This is often used in lists that list the important aspects of a location.
    \secretitem{Secret Item:} You can make the keyword red and italic by using the \texttt{\textbackslash secretitem\{...\}} command. This is often used to list a secret or hidden feature of a location.
    \coloritem[blue]{Color Item:}  You can make the keyword a custom color by using the \texttt{\textbackslash coloritem[color]\{...\}} command and specify the color in the square bracket.
\end{itemize}
\chapter{Gameplay Ability}
\begin{lstlisting}[style=styleCXX]

\end{lstlisting}
\end{segment}
\end{document}
